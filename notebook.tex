% Created 2019-04-30 Tue 20:34
% Intended LaTeX compiler: pdflatex
\documentclass[11pt]{article}
\usepackage[utf8]{inputenc}
\usepackage[T1]{fontenc}
\usepackage{graphicx}
\usepackage{grffile}
\usepackage{longtable}
\usepackage{wrapfig}
\usepackage{rotating}
\usepackage[normalem]{ulem}
\usepackage{amsmath}
\usepackage{textcomp}
\usepackage{amssymb}
\usepackage{capt-of}
\usepackage{hyperref}
\usepackage{minted}
\usepackage[margin=2cm]{geometry}
\usepackage{enumitem}
\DeclareMathOperator{\sign}{sign}
\setlength{\parindent}{0cm}
\date{\today}
\title{}
\hypersetup{
 pdfauthor={},
 pdftitle={},
 pdfkeywords={},
 pdfsubject={},
 pdfcreator={Emacs 26.2 (Org mode 9.2.3)}, 
 pdflang={English}}
\begin{document}


\section{Linguagens regulares}
\label{sec:orgb5c1d28}
\subsection{Alfabetos}
\label{sec:orge85a280}
Um alfabeto é denotado por \(\Sigma\). Exemplos:
\[
  \Sigma = \{\, 0, 1 \,\} \qquad
  \Sigma = \{\, \text{a}, \text{b}, \text{c}, \text{d}, \text{e} \,\} \qquad
  \Sigma = \{\, \triangle, \text{O}, \square, \text{X} \,\}
\]
\subsection{Palavras}
\label{sec:orgbe38543}
Uma palavra (ou cadeia) é uma sequência de zero ou mais símbolos do alfabeto.
\\[5pt]
\textbf{Notação}:
\begin{align*}
  & \lambda = \varnothing \\
  & 0^4 = 0000 \\
  & \Sigma^3 = \{ 000, 001, 010, 011, 100, 101, 110, 111 \} \\
  & \Sigma^* = \bigcup_{i \in \mathbb{N}} \Sigma^i \quad \text{conjunto de todas as possíveis palavras deste alfabeto.}
\end{align*}
\textbf{Concatenação}:
\begin{gather*}
  x = 00 \qquad y = 11 \\
  xy = 0011
\end{gather*}
\textbf{Reverso}:
\[
  (xy)^{\text{R}} = 1100
\]
Observação: uma palavra \(w\) é um palíndromo se, e somente se \(w^{\text{R}} = w\).
\subsection{Linguagens}
\label{sec:org121072a}
Uma linguagem é um conjunto de palavras \(L \subseteq \Sigma^*\). \\[5pt]
\textbf{Operações}:
\[ L_1L_2 = \{\, xy \,\mid\, x \in L_1,\, y \in L_2 \,\} \\ \]
\begin{align*}
  & L^0 = \{\, \lambda \,\} \\
  & L^1 = L \\
  & L^2 = LL \\
  & L^* = \bigcup_{i \in \mathbb{N}} L^i \quad \text{Fecho de Kleene} \\
  & L^+ = \bigcup_{i \in \mathbb{N}^*} L^i \\
  & \varnothing^* = \{\, \lambda \,\} \\
  & \varnothing^+ = \varnothing
\end{align*}
\textbf{Teorema}: As linguagens regulares são fechadas sob as seguintes operações
\begin{itemize}[itemsep=0pt]
\item União
\item Interseção
\item Complemento
\item Concatenação
\item Fecho de Kleene
\end{itemize}
\section{Autômatos finitos}
\label{sec:org74611e1}
\subsection{Determinísticos}
\label{sec:org4ca9476}
Um autômato finito determinístico é definido por:
\begin{align*}
  & Q && \text{Um conjunto finito de estados.} \\
  & \Sigma && \text{Um alfabeto finito.} \\
  & \delta: Q \times \Sigma \to Q && \text{Uma função de transição.} \\
  & q_o \in Q && \text{Um estado inicial.} \\
  & F \subseteq Q && \text{Um conjunto de estados finais.}
\end{align*}
\textbf{Notação}:
\begin{align*}
  & L(M) = A \qquad \text{A linguagem reconhecida pelo autômato $M$.} \\[5pt]
  & L(M: F = \varnothing) = \varnothing \\[5pt]
  & \hat{\delta}: Q \times \Sigma^* \to Q \\
  & \hat{\delta}(e, w): \text{aplicação sucessiva de }\delta\text{ aos símbolos de }w.
\end{align*}
Ainda, nos autômatos existe um estado especial, denonimado \(\emptyset\), que aprisiona
todas as transições omitidas.
\subsubsection{Computação}
\label{sec:org65eeb75}
Seja \(M = (Q,\, \Sigma,\, \delta,\, q_0,\, F)\) um autômato finito determinístico, e \(w
    \in \Sigma^*\). \\
Dizemos que \(M\) aceita \(w\) se existe uma \textbf{sequência} de estados
\(r_1, \,\hdots,\, r_n \in Q\) satisfazendo:
\begin{enumerate}
\item \(r_0 = q_0\)
\item \(\forall\, i \in [0, n): \delta(r_i,\, w_{i + 1}) = r_{i + 1}\)
\item \(r_n \in F\)
\end{enumerate}
Um autômato \(M\) reconhece uma linguagem \(L\) se \(\forall\, w \in L: M \text{ aceita } w\). \\
Uma linguagem é regular se existe um autômato finito que a reconhece.
\subsubsection{Minimização de estados}
\label{sec:org9ac2c25}
Dois estados \(e\) e \(e'\) são \textbf{equivalentes} se
\[
  \hat{\delta}(e, w) \in F \iff \hat{\delta}(e', w) \in F
\]
O algoritmo de minimização, então, é:
\begin{enumerate}
\item Produza uma partição \(P_0 = \{F,\, Q - F\}\) de \(Q\), separando os estados finais dos
não finais.
\item Para cada bloco de estados \(B\) na partição \(P_i\), cada símbolo \(s\) do
alfabeto \(\Sigma\), e cada par de estados (\(e\), \(e'\)) contidos no bloco B:
\begin{enumerate}
\item Sejam \(d = \delta(e, s)\) e \(d' = \delta(e' , s)\) os estados para os quais o AFD
transita quando lê o sı́mbolo \(s\) a partir dos estados \(e\) e \(e'\), respectivamente.
\item Se \(d\) e \(d'\) pertencem a blocos diferentes na partiçãoo \(P_i\), então
os estados \(e\) e \(e'\) não são equivalentes, e devem ser separados na partição
\(P_{i+1}\).
\end{enumerate}
\item Se a partição \(P_{i+1}\) for diferente da partiçãoo \(P_i\),
repita o passo 2.
\item O autômato mínimo é construído de tal forma que seus estados são os blocos da
última partição \(P\) produzida.
\end{enumerate}
\subsection{Não determinísticos}
\label{sec:org018ea35}
Um autômato finito não determinístico é definido por:
\begin{align*}
  & Q && \text{Um conjunto finito de estados.} \\
  & \Sigma && \text{Um alfabeto finito.} \\
  & \delta: Q \times \Sigma \to \mathcal{P}(Q) && \text{Uma função de transição.} \\
  & I \subseteq Q && \text{Um conjunto de estados iniciais.} \\
  & F \subseteq Q && \text{Um conjunto de estados finais.}
\end{align*}
Sendo \(a \in \Sigma\) um símbolo, e \(w \in \Sigma^*\) uma palavra, define-se a função de
transição estendida:
\begin{align*}
  & \hat{\delta}: Q \times \Sigma^* \to \mathcal{P}(Q) \\
  & \hat{\delta}(\emptyset,\, w) = \{\emptyset\} \\[5pt]
  & \hat{\delta}(X,\, \lambda) = X \\[5pt]
  & \hat{\delta}(X,\, aw) = \hat{\delta}\left(\,\bigcup_{l \in X} \delta(l,\, a),\, w \right)
\end{align*}
\textbf{Teorema}: Todo AFN possui um AFD equivalente. \\
Por construção:
\begin{align*}
  & Q = \mathcal{P}(Q_{\text{afn}}) && \\
  & \Sigma = \Sigma_{\text{afn}} && \\
  & \delta(X, a) = \bigcup_{l \in X} \delta_{\text{afn}}(l,\, a) && \\
  & q_o = I_{\text{afn}} && \\
  & F = \left\{ X \subseteq Q_{\text{afn}} \,\mid\, X \cap F \neq \varnothing \right\}&&
\end{align*}
\subsubsection{Transições \(\lambda\)}
\label{sec:orgf20752b}
Um autômato finito não determinístico com transições \(\lambda\) introduz a
possibilidade de transições sem a consumação de símbolos.
\begin{align*}
  & Q = Q_{\text{afn}} && \\
  & \Sigma = \Sigma_{\text{afn}} && \\
  & \delta: Q \times \Sigma_{\lambda} \to \mathcal{P}(Q) && \\
  & I = I_{\text{afn}} && \\
  & F = F_{\text{afn}} &&
\end{align*}
Onde \(\Sigma_{\lambda} = \Sigma \cup \{\lambda\}\). \\[10pt]
Os estados para os quais se transita sem consumir símbolos é definido pelo fecho \(\lambda\):
\[
  \mathcal{F}_{\lambda}: \mathcal{P}(Q) \to \mathcal{P}(Q)
\]
\textbf{Teorema}: O fecho lambda de um estado é pelo menos o próprio estado.
\[
  \forall\, X \in Q: X \in \mathcal{F}_{\lambda}(\{X\})
\]
Assim, define-se a função de transição estendida:
\begin{align*}
  & \hat{\delta}: Q \times \Sigma_{\lambda}^* \to \mathcal{P}(Q) \\
  & \hat{\delta}(\varnothing, w) = \varnothing \\
  & \hat{\delta}(X, \lambda) = \mathcal{F}_{\lambda}(X) \\
  & \hat{\delta}(X, ay) = \hat{\delta} \left( \bigcup_{Y \in\, \mathcal{F}_{\lambda}(X)} \delta(Y, a),\enspace y \right)
\end{align*}
\textbf{Teorema}: Todo AFN\(\lambda\) possui um AFN equivalente. \\
Por construção:
\begin{align*}
  & Q = Q_{\text{afn}\lambda} && \\
  & \Sigma = \Sigma_{\text{afn}\lambda} && \\
  & \delta = \mathcal{F}_{\lambda} \circ \delta_{\text{afn}\lambda} && \\
  & I = \mathcal{F}_{\lambda}\left(I_{\text{afn}\lambda}\right) && \\
  & F = F_{\text{afn}\lambda} &&
\end{align*}
\section{Expressões regulares}
\label{sec:org7329733}
Uma expressão regular pode ser uma das seguintes formas, cada qual com a linguagem
correspondente:
\begin{align*}
  & \lambda & \{\lambda\} && \\
  & \varnothing & \varnothing && \\
  & a & \{a\} && \\
  & R_1 + R_2 & L(R_1) \cup L(R_2)  && \\
  & R_1 R_2 & L(R_1) \cdot L(R_2)  && \\
  & R^* & L(R)^* 
\end{align*}
\textbf{Operações}:
\begin{align*}
  & R^+ = RR^* && \\
  & R^0 = \lambda && \\
  & R^n = RR^{(n - 1)} &&
\end{align*}
\section{Linguagens irregulares}
\label{sec:org25ae287}
Nas linguagens regulares, têm se o \textbf{lema do bombardeamento}: \\
Se \(L\) é uma linguagem regular, então
\begin{align*}
  & \exists\, k \in \mathbb{N}^*: \\
  & \quad \forall\, z \in L, |z| \geq k : \\
  & \quad\quad \exists\, u, v, w: \\
  & \quad\quad\quad 1.\> z = uvw \\
  & \quad\quad\quad 2.\> |uv| \leq k \\
  & \quad\quad\quad 3.\> v \neq \lambda \\
  & \quad\quad\quad 4.\> \forall\,i \in \mathbb{N}^*: \left(uv^iw\right) \in L
\end{align*}
O lema pode ser utilizado para provar que uma dada linguagem não é regular.
\section{Linguagens livres de contexto}
\label{sec:orgd65c2c1}
Uma linguagem livre de contexto é uma linguagem que pode ser denotada por uma gramática
livre de contexto.
\subsection{Gramáticas livres de contexto}
\label{sec:org2e9da09}
Uma gramática livre de contexto é definida por:
\begin{align*}
  & V && \text{Um conjunto finito de variáveis.} \\
  & \Sigma && \text{Um alfabeto finito.} \\
  & R && \text{Um conjunto de regras.} \\
  & S \in V && \text{Uma variável inicial.}
\end{align*}
As regras são constituídas da seguinte forma:
\begin{enumerate}
\item O lado esquerdo de uma regra é constituído por uma única variável.
\item O lado direito é constituído por uma combinação de terminais e variáveis.
\end{enumerate}
Por convenção a variável inicial é a variável alvo da primeira regra. \\

Exemplo:
\begin{align*}
  G = \big(\{A, B\}, & \, \{0, 1, 5\},\, R,\, A \big) \\
  R: \quad & A \to 0A1 \\
           & A \to B \\
           & B \to 5
\end{align*}
\end{document}
