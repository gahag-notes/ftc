% Created 2019-03-21 Thu 20:32
% Intended LaTeX compiler: pdflatex
\documentclass[11pt]{article}
\usepackage[utf8]{inputenc}
\usepackage[T1]{fontenc}
\usepackage{graphicx}
\usepackage{grffile}
\usepackage{longtable}
\usepackage{wrapfig}
\usepackage{rotating}
\usepackage[normalem]{ulem}
\usepackage{amsmath}
\usepackage{textcomp}
\usepackage{amssymb}
\usepackage{capt-of}
\usepackage{hyperref}
\usepackage{minted}
\usepackage[margin=2cm]{geometry}
\DeclareMathOperator{\sign}{sign}
\date{\today}
\title{}
\hypersetup{
 pdfauthor={},
 pdftitle={},
 pdfkeywords={},
 pdfsubject={},
 pdfcreator={Emacs 26.1 (Org mode 9.2.2)}, 
 pdflang={English}}
\begin{document}


\section{Linguagens formais}
\label{sec:org4bee8ec}
\(\Sigma\): alfabeto \\
Exemplos:
\begin{align*}
  & \Sigma = \{\, 0, 1 \,\} \\
  & \Sigma = \{\, \text{a}, \text{b}, \text{c}, \text{d}, \text{e} \,\} \\
  & \Sigma = \{\, \triangle, \text{O}, \square, \text{X} \,\} \\
\end{align*}
Palavra (cadeia) é uma sequência de \(0\) ou mais símbolos do alfabeto. \\
\(\lambda\) é a palavra vazia. \\
\(|w|\) denota o tamanho da palavra \(w\), i.e. o número de símbolos na palavra. \\
\(\Sigma^*\) é o conjunto de todas as possíveis palavras constituídas de símbolos
deste alfabeto. \\
Notação:
\begin{align*}
  & 0^4 = 0000 \\
  & \Sigma^3 = \{ 000, 001, 010, 011, 100, 101, 110, 111 \}
\end{align*}
Uma linguagem é um conjunto de palavras \(L \subseteq \Sigma^*\). \\
Como uma linguagem é um conjunto, as operações sobre conjuntos se aplicam. \\
\subsection{Operações}
\label{sec:orgc8ad2f0}
Concatenação:
\begin{align*}
  & x = 00 \\
  & y = 11 \\
  & xy = 0011
\end{align*}
Reverso:
\[
  (xy)^{\text{R}} = 1100
\]
Observação: uma palavra \(w\) é um palíndromo se, e somente se \(w^{\text{R}} = w\). \\
Em linguagens:
\[ L_1L_2 = \{\, xy \,\mid\, x \in L_1,\, y \in L_2 \,\} \\ \]
\begin{align*}
  & L^0 = \{\, \lambda \,\} \\
  & L^1 = L \\
  & L^2 = LL \\
  & L^* = \bigcup_{i \in \mathbb{N}} L^i \quad \text{Fecho de Kleene} \\
  & L^+ = \bigcup_{i \in \mathbb{N}^*} L^i \\
  & \emptyset^* = \{\, \lambda \,\} \\
  & \emptyset^+ = \emptyset
\end{align*}
\end{document}
