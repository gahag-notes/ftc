% Created 2019-03-28 Thu 20:23
% Intended LaTeX compiler: pdflatex
\documentclass[11pt]{article}
\usepackage[utf8]{inputenc}
\usepackage[T1]{fontenc}
\usepackage{graphicx}
\usepackage{grffile}
\usepackage{longtable}
\usepackage{wrapfig}
\usepackage{rotating}
\usepackage[normalem]{ulem}
\usepackage{amsmath}
\usepackage{textcomp}
\usepackage{amssymb}
\usepackage{capt-of}
\usepackage{hyperref}
\usepackage{minted}
\usepackage[margin=2cm]{geometry}
\DeclareMathOperator{\sign}{sign}
\date{\today}
\title{}
\hypersetup{
 pdfauthor={},
 pdftitle={},
 pdfkeywords={},
 pdfsubject={},
 pdfcreator={Emacs 26.1 (Org mode 9.2.2)}, 
 pdflang={English}}
\begin{document}


\section{Linguagens formais}
\label{sec:org700ee4f}
\subsection{Alfabetos}
\label{sec:orgd0d5410}
Um alfabeto é denotado por \(\Sigma\). Exemplos:
\begin{align*}
  & \Sigma = \{\, 0, 1 \,\} \\
  & \Sigma = \{\, \text{a}, \text{b}, \text{c}, \text{d}, \text{e} \,\} \\
  & \Sigma = \{\, \triangle, \text{O}, \square, \text{X} \,\}
\end{align*}
\subsection{Palavras}
\label{sec:org06083bf}
Uma palavra (ou cadeia) é uma sequência de zero ou mais símbolos do alfabeto.
\\[5pt]
Notação:
\begin{align*}
  & \lambda = \varnothing \\
  & 0^4 = 0000 \\
  & \Sigma^3 = \{ 000, 001, 010, 011, 100, 101, 110, 111 \} \\
  & \Sigma^* = \bigcup_{i \in \mathbb{N}} \Sigma^i \quad \text{conjunto de todas as possíveis palavras deste alfabeto.}
\end{align*}
Concatenação:
\begin{align*}
  & x = 00 \\
  & y = 11 \\
  & xy = 0011
\end{align*}
Reverso:
\[
  (xy)^{\text{R}} = 1100
\]
Observação: uma palavra \(w\) é um palíndromo se, e somente se \(w^{\text{R}} = w\).
\subsection{Linguagens}
\label{sec:org04ab6ee}
Uma linguagem é um conjunto de palavras \(L \subseteq \Sigma^*\).
\subsubsection{Operações}
\label{sec:orgaf180bd}
\[ L_1L_2 = \{\, xy \,\mid\, x \in L_1,\, y \in L_2 \,\} \\ \]
\begin{align*}
  & L^0 = \{\, \lambda \,\} \\
  & L^1 = L \\
  & L^2 = LL \\
  & L^* = \bigcup_{i \in \mathbb{N}} L^i \quad \text{Fecho de Kleene} \\
  & L^+ = \bigcup_{i \in \mathbb{N}^*} L^i \\
  & \varnothing^* = \{\, \lambda \,\} \\
  & \varnothing^+ = \varnothing
\end{align*}
\subsection{Autômatos}
\label{sec:orgaefb11e}
Um autômato finito determinístico é definido por:
\begin{align*}
  & Q && \text{Um conjunto finito de estados.} \\
  & \Sigma && \text{Um alfabeto finito.} \\
  & \delta: Q \times \Sigma \to Q && \text{Uma função de transição.} \\
  & q_o \in Q && \text{Um estado inicial.} \\
  & F \subseteq Q && \text{Um conjunto de estados finais.}
\end{align*}
Notação:
\begin{align*}
  & L(M) = A \qquad \text{A linguagem reconhecida pelo autômato $M$.} \\[5pt]
  & L(M: F = \varnothing) = \varnothing \\[5pt]
  & \hat{\delta}(e, w): \text{aplicação sucessiva de }\delta\text{ aos símbolos de }w.
\end{align*}
\subsubsection{Computação}
\label{sec:orgf450fc4}
Seja \(M = (Q,\, \Sigma,\, \delta,\, q_0,\, F)\) um autômato finito, e \(w \in \Sigma^*\).
Dizemos que \(M\) aceita \(w\) se existe uma \textbf{sequência} de estados
\(r_1, \,\hdots,\, r_n \in Q\) satisfazendo:
\begin{enumerate}
\item \(r_0 = q_0\)
\item \(\forall\, i \in [0, n): \delta(r_i,\, w_{i + 1}) = r_{i + 1}\)
\item \(r_n \in F\)
\end{enumerate}
Um autômato \(M\) reconhece uma linguagem \(L\) se \(\forall\, w \in L: M \text{ aceita } w\). \\
Uma linguagem é regular se existe um autômato finito que a reconhece.
\subsubsection{Minimização de estados}
\label{sec:orgd357e0e}
Dois estados \(e\) e \(e'\) são \textbf{equivalentes} se
\[
  \hat{\delta}(e, w) \in F \iff \hat{\delta}(e', w) \in F
\]
O algoritmo de minimização, então, é:
\begin{enumerate}
\item Produza uma partição \(\mathcal{P}_0 = \{F,\, Q - F\}\) do conjunto de estados \(Q\),
em que os estados finais estão separados dos não finais.
\item Para cada bloco de estados \(B\) na partição \(\mathcal{P}_i\), cada símbolo \(s\) do
alfabeto \(\Sigma\), e cada par de estados (\(e\), \(e'\)) contidos no bloco B:
\begin{enumerate}
\item Sejam \(d = \delta(e, s)\) e \(d' = \delta(e' , s)\) os estados para os quais o AFD
transita quando lê o sı́mbolo \(s\) a partir dos estados \(e\) e \(e'\), respectivamente.
\item Se \(d\) e \(d'\) pertencem a blocos diferentes na partiçãoo \(\mathcal{P}_i\), então
os estados \(e\) e \(e'\) não são equivalentes, e devem ser separados na partição
\(\mathcal{P}_{i+1}\).
\end{enumerate}
\item Se a partição \(\mathcal{P}_{i+1}\) for diferente da partiçãoo \(\mathcal{P}_i\),
repita o passo 2.
\item O autômato mínimo é construído de tal forma que seus estados são os blocos da
última partição \(\mathcal{P}\) produzida.
\end{enumerate}
\end{document}
